В процессе компиляции программы обычно используется \textbf{промежуточное представление программы (IR)}, оно допускает
наличие неограниченного числа переменных. Однако реальная архитектура процессора предоставляет ограниченное
количество регистров. Это ставит перед компилятором задачу корректного распределения переменных по имеющимся
регистрам, при нехватке регистров необходимо некоторые переменные разместить в памяти. % Показать в какой конкретно?

Для понимания задачи рассмотрим пример: программа содержит выражение, в котором используются 3 переменные,
в то время как архитектура процессора предусматривает лишь 2 регистра.
В этом случае однозначно распределить переменные по регистрам не получится.
Возникает необходимость проводить \textbf{выгрузку} (в англоязычной литературе \textit{spill})
какой-то из переменных в память. При обращении к этой переменной потребуется загрузить
её из памяти, а при изменении — снова сохранить обратно.

Таким образом, важно не только обеспечить корректность выполнения программы, но и минимизировать количество обращений к памяти,
поскольку в настоящее время скорость работы с регистрами может отличаться от скорости работы с внешней памятью на порядки.