В процессе компиляции программы обычно используется \textbf{промежуточное представление программы (IR)}, которое допускает
наличие неограниченного числа переменных. Однако реальная архитектура процессора предоставляет ограниченное
количество регистров. Это ставит перед компилятором задачу корректного распределения переменных по имеющимся
регистрам, при нехватке регистров необходимо размещать некоторые переменные в памяти. % Показать, в какой именно?

Для понимания задачи рассмотрим пример: программа содержит выражение, в котором используются три переменные,
в то время как архитектура процессора предусматривает лишь два регистра.
В этом случае однозначно распределить переменные по регистрам не получится.
Возникает необходимость проводить \textbf{выгрузку} (в англоязычной литературе \textit{spill})
одной из переменных в память. При обращении к этой переменной потребуется загрузить
её из памяти, а при изменении — снова сохранить обратно.

Таким образом, важно не только обеспечить корректность выполнения программы, но и минимизировать количество обращений к памяти,
поскольку в настоящее время скорость работы с регистрами превышает скорость работы с внешней памятью на один-два порядка.
