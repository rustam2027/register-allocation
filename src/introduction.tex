В процессе компиляции программы в машинный код конкретного процессора возникает проблема \textbf{распределения регистров}. % Написать конкретно когда (Во время перехода из IR в ASM?)
Множество исследований посвящено данной проблеме, мы же рассмотрим решения предложенные в
работах Чайтина (Chatin)\cite{chaitin1982}, Бриггса (Briggs)\cite{briggs1994}, Полетто (Poletto)\cite{poletto1999}. % Добавить ссылки на работы
Отметим что задача является NP полной (сложность задачи будет рассмотрена в секции~\ref{seg:complexity}),
и поэтому все алгоритмы, описанные ниже, являются эвристическими.

В рассматриваемых работах предложено два подхода к решению этой проблемы.
Первый, через построение \textbf{графа}, этот подход был предложен еще до работ Чайтина в работах Кокка
(Cocke) в 1970\cite{cocke1970}, Ершова в 1962\cite{ershov1962}, Шварца (Schwartz) в 1973\cite{schwartz1973}. % На самом деле в своей работе чайтин говорит что до этого не было имплементации
Второй подход это \textbf{линейная аллокация}, который для распределения регистров использует не граф,
а линейное представление кода. Этот метод был предложен в работе Полетто.