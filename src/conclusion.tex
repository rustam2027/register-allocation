Как следует из приведённых выше сравнений, линейная аллокация выполняется быстрее, чем алгоритмы,
основанные на раскраске графов.
Однако скорость исполнения программ, скомпилированных с использованием
этого алгоритма, незначительно отличается от скорости программ,
скомпилированных с использованием алгоритмов,
основанных на раскраске графов.

Кроме того, сам алгоритм значительно проще в реализации.

Совокупность этих факторов делает его привлекательной альтернативой стандартным методам аллокации регистров там,
где важна скорость компиляции.
Примером таких областей применения является, например, JIT-компиляция.
