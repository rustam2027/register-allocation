\subsection{Анализ проблем}

В своей статье Бриггс описывает модифицированный алгоритм раскраски графа, он называет его \textbf{optimistic coloring}.
Разберём какие именно проблемы скрываются в вышеописанных примерах, и как можно их решить.

В примере, представленном на рисунке~\ref{fig:ex2}, как было отмечено ранее, алгоритм Чайтина производит выгрузку хотя бы одной переменной.
Однако выгрузку можно избежать.
Эта проблема связана с тем, что по умолчанию алгоритм считает, что если в графе нет вершин со степенью меньшей $N$,
то такой граф нельзя раскрасить в $N$ цветов.

% В примере с SVD разложением проблема возникает не когда алгоритм выбирает сгрузить переменные $i$ и $n$, а когда
% после выгрузки $N$ и $M$ он не обнаруживает что переменные $n$ и $i$ могут поместиться на регистры.

В примере с SVD-разложением проблема возникает не когда алгоритм выбирает выгрузить переменные $i$ и $n$, а когда
после выгрузки $N$ и $M$ он не обнаруживает, что переменные $n$ и $i$ могут поместиться на регистры.
Во время этапа~\ref{chatin_algo_color_assignment} алгоритм не рассматривает причину выгрузки
переменных $i$ и $n$.
Он не понимает, что причиной для выгрузки $i$ и $n$ были $I$ и $N$.
Однако переменные $I$ и $N$ были выгружены,
а значит, $i$ и $n$ могут быть размещены на регистрах.

\subsection{Идея}

Для решения вышеперечисленных проблем Бриггс предлагает внести следующие корректировки в алгоритм Чайтина:

\begin{enumerate}
    \item В пункте~\ref{chatin_algo_spill} алгоритма, когда все оставшиеся вершины имеют не менее $N$ соседей,
    по тому же принципу выберем переменную для выгрузки, однако вместо немедленной выгрузки, разместим ее на стек. \label{briggs_change_spill}

    \item Соответственно, в пункте \ref{chatin_algo_color_assignment} возникает проблема, ведь теперь не всегда
    вершину можно будет покрасить.
    Если оказалось, что для вершины нельзя подобрать цвет, оставим ее неокрашенной.
    Непокрашенные переменные --- это переменные, которые нужно выгрузить.
    
\end{enumerate}

Эти изменения помогут решить обнаруженные проблемы алгоритма Чайтина. Первая проблема решается в пункте \ref{briggs_change_spill}.
Теперь в примере, представленном на рисунке~\ref{fig:ex2}, хотя некоторая вершина и будет выбрана для выгрузки, на этапе \ref{chatin_algo_color_assignment}
всем вершинам удастся получить цвет.

Вторая проблема также решена. Теперь хотя переменные $i$, $n$, $M$, $N$ и будут выгружены (можно считать что в таком
порядке), при попытке подобрать цвета для $M$ и $N$ станет понятно, что подобрать для них цвета не получится.
Поэтому переменные $i$ и $n$ не придется выгружать.

Эвристика предыдущего алгоритма скорее отвечала на
вопрос ``имеет ли вершина менее $N$ соседей?'', чем на вопрос ``можно ли подобрать цвет для этой вершины?''.
Новый алгоритм стремится ответить на последний вопрос.
